\chapter{Processazione pezzi chirurgici e bioptici}

\section{Introduzione}
Dopo la rimozione di un campione di tessuto dal paziente, una serie di processi fisici e chimici deve essere effettuata per garantire che i preparati microscopici finali siano di qualità diagnostica. I tessuti vengono esposti a una sequenza di reagenti che li fissano, disidratano, chiarificano e infiltrano. Infine, il tessuto viene inglobato in un mezzo che ne fornisce supporto per il microtomo. La qualità della conservazione strutturale dei componenti del tessuto è determinata dalla scelta dei tempi di esposizione ai reagenti durante la processazione. Ogni fase della processazionedel tessuto è cruciale, dalla selezione del campione alla determinazione dei protocolli e reagenti appropriati, fino alla colorazione e diagnosi finale. La produzione di preparati di qualità per la diagnosi richiede competenze che si sviluppano con la pratica continua e l'esperienza. Con lo sviluppo di nuove tecnologie e strumentazioni, il ruolo del laboratorio di istologia nell'assistenza ai pazienti continuerà ad evolvere, migliorando la standardizzazione dei processi, aumentando la produttività e ottimizzando l'uso delle risorse disponibili. Questo capitolo fornirà una panoramica dei passaggi necessari e dei reagenti utilizzati per preparare i tessuti alla valutazione microscopica.

\section{Etichettatura dei Tessuti}
A ciascun campione di tessuto deve essere assegnato un numero o codice di accesso univoco, come discusso nel Capitolo 5. Questo codice deve accompagnare i campioni durante tutto il processo in laboratorio e può essere generato manualmente o elettronicamente. Le nuove tecnologie hanno reso disponibili in molti laboratori sistemi di riconoscimento a codice a barre, risposta rapida (QR) e caratteri. Sistemi di pre-etichettatura automatizzati, che incidono o imprimono permanentemente cassette e vetrini, insieme a penne e matite resistenti ai reagenti chimici, sono utilizzati di routine nei laboratori di patologia. Indipendentemente dal fatto che si usi un sistema manuale o automatizzato, devono essere implementate procedure e politiche adeguate per garantire l'identificazione positiva dei blocchi di tessuto e dei vetrini durante la processazione, la diagnosi e l'archiviazione.

\section{Principi dela processazione dei Tessuti}
la processazione dei tessuti è finalizzato alla rimozione di tutta l'acqua estraibile dal tessuto, sostituendola con un mezzo di supporto che fornisce sufficiente rigidità per consentire il sezionamento del tessuto senza danneggiare o distorcere il parenchima.

\subsection{Fattori che influenzano la velocità di processazione}
Quando il tessuto è immerso in un fluido, si verifica uno scambio tra il liquido presente nel tessuto e il fluido circostante. La velocità di questo scambio dipende dalla superficie del tessuto esposta ai reagenti di processazione. Diversi fattori influenzano la velocità con cui avviene l'intercambio: agitazione, calore, viscosità e vuoto.

\subsubsection{Agitazione}
L'agitazione aumenta il flusso di soluzioni fresche intorno al tessuto. I processatori automatici utilizzano meccanismi di oscillazione verticale o rotativa o la rimozione e sostituzione pressurizzata dei fluidi a intervalli di tempo per agitare i campioni. Un'agitazione efficiente può ridurre il tempo totale di processazione fino al 30\%.

\subsubsection{Calore}
Il calore accelera la penetrazione e lo scambio di fluidi. Tuttavia, il calore deve essere usato con parsimonia per ridurre il rischio di restringimento, indurimento o fragilità del campione. Temperature superiori ai 45°C possono essere dannose per le successive tecniche di immunoistochimica.

\subsubsection{Viscosità}
La viscosità è la proprietà che misura la resistenza di un fluido al flusso. Soluzioni con molecole di dimensioni più piccole presentano una penetrazione più rapida (bassa viscosità), mentre molecole più grandi rallentano lo scambio (alta viscosità). La maggior parte dei reagenti usati nela processazione, nella disidratazione e nello schiarimento hanno viscosità simili, tranne l'olio di cedro.

\subsubsection{Vuoto}
L'uso della pressione per aumentare il tasso di infiltrazione riduce il tempo necessario per completare ogni fase dela processazione dei campioni. Il vuoto, se applicato, non deve superare i 50,79 kPa per evitare danni ai tessuti. Può essere utile anche per rimuovere l'aria intrappolata nei tessuti porosi, accelerando l'impregnazione di tessuti densi e grassi.

\section{Fasi dela processazione dei Tessuti}
\begin{itemize}
    \item \textbf{Fissazione}: stabilizza e indurisce il tessuto con una distorsione minima delle cellule.
    \item \textbf{Disidratazione}: rimozione dell'acqua e del fissativo dal tessuto.
    \item \textbf{Schiarimento}: eliminazione delle soluzioni disidratanti per rendere i componenti del tessuto ricettivi al mezzo infiltrante.
    \item \textbf{Infiltrazione}: impregnazione del tessuto con un mezzo di supporto.
    \item \textbf{Inclusione}: orientamento del campione nel mezzo di supporto e solidificazione del preparato.
\end{itemize}

\section{Fissazione}
Preservare le cellule e i componenti tessutali con la minima distorsione possibile è l'obiettivo principale dela processazione dei campioni di tessuto. La fissazione stabilizza le proteine, rendendo la cellula e i suoi componenti resistenti all'autolisi. È fondamentale che la fissazione sia completata prima di iniziare i passaggi successivi.

\section{Disidratazione}
La disidratazione consiste nella rimozione dell'acqua libera e dei fissativi acquosi dai componenti del tessuto. La disidratazione deve essere eseguita gradualmente attraverso una serie di soluzioni a concentrazione crescente. L'etanolo è uno dei reagenti disidratanti più comuni.

\section{Chiarificazione}
I reagenti schiarenti agiscono come intermediari tra i disidratanti e i mezzi di infiltrazione. Devono essere miscibili con entrambi. Il reagente di schiarimento più comunemente usato nei laboratori di istologia è lo xilene.

\section{Mezzi di inclusione alternativi}
In alcuni casi, la paraffina può risultare inadatta per l'inclusione di determinati tipi di tessuti, come nei seguenti esempi:
\begin{itemize}
    \item I reagenti di processazione rimuovono o distruggono i componenti tissutali oggetto dell'indagine, come i lipidi;
    \item È necessario ottenere sezioni più sottili, ad esempio nei linfonodi;
    \item L'uso del calore può danneggiare i tessuti o gli enzimi;
    \item Il mezzo di infiltrazione non è sufficientemente rigido per supportare il tessuto.
\end{itemize}

\subsection{Resina}
La resina è utilizzata esclusivamente come mezzo di inclusione per la microscopia elettronica (vedi Capitolo 22), in particolare per ottenere sezioni ultrafini ad alta risoluzione e per sezioni di osso non decalcificato.

\subsection{Agar}
Il gel di agar, da solo, non fornisce un supporto sufficiente per sezionare i tessuti. Il suo uso principale è come agente coesivo per piccoli frammenti di tessuto friabili dopo la fissazione, in un processo noto come "doppia inclusione". I frammenti di tessuto vengono immersi in agar fuso, lasciati solidificare e poi tagliati per la processazione di routine. Un metodo superiore e più raffinato consiste nel filtrare il fissativo contenente i frammenti di tessuto attraverso un filtro Millipore con l'ausilio di una pompa di aspirazione. L'agar fuso viene quindi versato con cura nel filtro e, una volta solidificato, il pellet di agar ottenuto viene processato e incluso in paraffina.

\subsection{Gelatina}
La gelatina è utilizzata principalmente per la produzione di sezioni di organi interi con la tecnica Gough-Wentworth e per le sezioni congelate. Il suo utilizzo è raro.

\subsection{Celloidina}
L'uso della celloidina o di LVN (nitrocellulosa a bassa viscosità) è sconsigliato a causa dei requisiti particolari necessari per ospitare i reagenti di processazione e per l'uso limitato che queste sezioni trovano in neuropatologia. L'uso della celloidina è quindi molto raro.

\section{Inclusione in paraffina}
L'inclusione prevede l'incapsulamento di campioni orientati correttamente, dopo essere stati adeguatamente processati, in un mezzo di supporto che fornisca stabilità durante il microtaglio. Il mezzo di inclusione deve riempire la matrice del tessuto, sostenendo i componenti cellulari. Dovrebbe anche fornire elasticità, resistenza alla distorsione durante il taglio, facilitando al contempo il processo di sezionamento.

La maggior parte dei laboratori utilizza centri di inclusione modulari, composti da un dispensatore di paraffina, una piastra fredda e un'area riscaldata per stampi e cassette tissutali. La paraffina viene automaticamente dispensata in uno stampo della dimensione appropriata. Il tessuto viene orientato nello stampo, si applica una cassetta che produce una superficie piana con lati paralleli, e lo stampo viene quindi raffreddato rapidamente, garantendo una struttura cristallina fine e minimizzando gli artefatti durante il taglio.

\subsection{Orientamento dei tessuti}
L'orientamento del campione durante l'inclusione è fondamentale per la corretta dimostrazione della morfologia. Un orientamento scorretto può causare danni agli elementi diagnostici o impedirne la visualizzazione al microscopio. Sono disponibili prodotti che aiutano a garantire un orientamento corretto, come sistemi di marcatura, coloranti tatuanti, sacchetti per biopsia, spugne e carte. L'orientamento deve offrire la minore resistenza possibile del tessuto contro il coltello durante il taglio. Un margine di mezzo di inclusione attorno al tessuto fornisce un ulteriore supporto.

\subsection{Tessuti che richiedono orientamento speciale}
Alcuni tessuti richiedono una particolare attenzione per quanto riguarda l'orientamento durante l'inclusione, tra cui:
\begin{itemize}
    \item Strutture tubulari: la sezione deve mostrare la parete e il lume, come nel caso di arterie, vene, tube di Falloppio e deferenti;
    \item Biopsie cutanee: per raschiamenti, punch o escissioni, è necessario che la sezione mostri l'epidermide, il derma e lo strato sottocutaneo;
    \item Biopsie intestinali e della colecisti: devono essere tagliate perpendicolarmente alla superficie, orientate in modo che l'epitelio sia l'ultimo strato a essere sezionato, minimizzando la compressione e la distorsione;
    \item Biopsie muscolari: devono contenere sezioni sia trasversali che longitudinali;
    \item Campioni multipli di tessuti devono essere orientati uno accanto all'altro con la superficie epiteliale rivolta nella stessa direzione.
\end{itemize}

\section{Processazione automatizzata dei tessuti}
Il principio di base per la processazione dei tessuti prevede lo scambio di fluidi utilizzando una serie di soluzioni per un periodo di tempo predeterminato in un ambiente controllato. Le attrezzature per la processazione dei tessuti sono rimaste relativamente invariate per decenni, ma i recenti progressi includono forni a microonde specializzati, processatori a throughput costante e processatori con retorti multi-sezione.

\subsection{Processatori a microonde}
I forni a microonde progettati specificamente per la processazione dei tessuti sono ora comuni. Questi strumenti riducono i tempi di processazione da ore a minuti, accelerando la diffusione delle soluzioni nei tessuti attraverso l'aumento del calore interno del campione. I forni a microonde da laboratorio hanno un controllo preciso della temperatura, dei tempi e dei sistemi di estrazione dei fumi. Tuttavia, questo metodo richiede attenzione per il controllo della temperatura, e i costi di tali forni possono risultare elevati.

\subsection{Vantaggi delle nuove tecnologie}
I vantaggi principali offerti dalle nuove tecnologie di processazione includono:
\begin{itemize}
    \item Programmi personalizzabili in base al tipo di tessuto;
    \item Schedulazione rapida;
    \item Contenimento dei fluidi e dei fumi;
    \item Reagenti ecologici.
\end{itemize}

\section{Riassunto}
La processazione dei campioni chirurgici e bioptici è un passaggio fondamentale per la preparazione dei tessuti alla valutazione microscopica. Ogni campione viene prima etichettato con un codice unico, garantendo la sua identificazione lungo tutte le fasi del processo. La fissazione stabilizza i tessuti, impedendone la degradazione, mentre la disidratazione elimina l'acqua. Successivamente, il chiarimento prepara i tessuti per l'infiltrazione, che consiste nell'impregnazione con un mezzo di supporto, solitamente paraffina, necessario per il sezionamento.La qualità del campione dipende da fattori come agitazione, calore, viscosità e vuoto, che influenzano la velocità di penetrazione dei reagenti nei tessuti. Alcuni campioni richiedono orientamenti specifici durante l'inclusione, per garantire una corretta visualizzazione microscopica. Le nuove tecnologie, come i processatori a microonde, hanno rivoluzionato il settore, riducendo i tempi di lavorazione e migliorando la qualità delle preparazioni. L'uso di strumenti automatizzati consente una standardizzazione maggiore, personalizzazione dei protocolli e un impatto ambientale ridotto, aumentando efficienza e affidabilità.
