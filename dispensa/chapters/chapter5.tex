\chapter{Processazione pezzi chirurgici e bioptici}

\section{Introduzione}
Dopo la rimozione di un campione di tessuto dal paziente, una serie di processi fisici e chimici deve essere effettuata per garantire che i preparati microscopici finali siano di qualità diagnostica. I tessuti vengono esposti a una sequenza di reagenti che li fissano, disidratano, chiarificano e infiltrano. Infine, il tessuto viene inglobato in un mezzo che ne fornisce supporto per il microtomo. La qualità della conservazione strutturale dei componenti del tessuto è determinata dalla scelta dei tempi di esposizione ai reagenti durante la processazione. Ogni fase della processazionedel tessuto è cruciale, dalla selezione del campione alla determinazione dei protocolli e reagenti appropriati, fino alla colorazione e diagnosi finale. La produzione di preparati di qualità per la diagnosi richiede competenze che si sviluppano con la pratica continua e l'esperienza. Con lo sviluppo di nuove tecnologie e strumentazioni, il ruolo del laboratorio di istologia nell'assistenza ai pazienti continuerà ad evolvere, migliorando la standardizzazione dei processi, aumentando la produttività e ottimizzando l'uso delle risorse disponibili. Questo capitolo fornirà una panoramica dei passaggi necessari e dei reagenti utilizzati per preparare i tessuti alla valutazione microscopica.

\section{Etichettatura dei Tessuti}
A ciascun campione di tessuto deve essere assegnato un numero o codice di accesso univoco, come discusso nel Capitolo 2. Questo codice deve accompagnare i campioni durante tutto il processo in laboratorio e può essere generato manualmente o elettronicamente. Le nuove tecnologie hanno reso disponibili in molti laboratori sistemi di riconoscimento a codice a barre, risposta rapida (QR) e caratteri. Sistemi di pre-etichettatura automatizzati, che incidono o imprimono permanentemente cassette e vetrini, insieme a penne e matite resistenti ai reagenti chimici, sono utilizzati di routine nei laboratori di patologia. Indipendentemente dal fatto che si usi un sistema manuale o automatizzato, devono essere implementate procedure e politiche adeguate per garantire l'identificazione positiva dei blocchi di tessuto e dei vetrini durante la processazione, la diagnosi e l'archiviazione.

\section{Principi dela processazione dei Tessuti}
la processazione dei tessuti è finalizzato alla rimozione di tutta l'acqua estraibile dal tessuto, sostituendola con un mezzo di supporto che fornisce sufficiente rigidità per consentire il sezionamento del tessuto senza danneggiare o distorcere il tessuto stesso.

\subsection{Fattori che influenzano la velocità di processazione}
Quando il tessuto è immerso in un fluido, si verifica uno scambio tra il liquido presente nel tessuto e il fluido circostante. La velocità di questo scambio dipende dalla superficie del tessuto esposta ai reagenti di processazione. Diversi fattori influenzano la velocità con cui avviene l'intercambio: agitazione, calore, viscosità e vuoto.

\subsubsection{Agitazione}
L'agitazione aumenta il flusso di soluzioni fresche intorno al tessuto. I processatori automatici utilizzano meccanismi di oscillazione verticale o rotativa o la rimozione e sostituzione pressurizzata dei fluidi a intervalli di tempo per agitare i campioni. Un'agitazione efficiente può ridurre il tempo totale di processazione fino al 30\%.

\subsubsection{Calore}
Il calore accelera la penetrazione e lo scambio di fluidi. Tuttavia, il calore deve essere usato con parsimonia per ridurre il rischio di restringimento, indurimento o fragilità del campione. Temperature superiori ai 45°C possono essere dannose per le successive tecniche di immunoistochimica.

\subsubsection{Viscosità}
La viscosità è la proprietà che misura la resistenza di un fluido al flusso. Soluzioni con molecole di dimensioni più piccole presentano una penetrazione più rapida (bassa viscosità), mentre molecole più grandi rallentano lo scambio (alta viscosità). La maggior parte dei reagenti usati nela processazione, nella disidratazione e nello schiarimento hanno viscosità simili, tranne l'olio di cedro. I materiali per l'inclusione hanno viscosità differente e la paraffina fusa ha una minore viscosità, aumentanto la velocità di imprengazione.

\subsubsection{Vuoto}
L'uso della pressione per aumentare il tasso di infiltrazione riduce il tempo necessario per completare ogni fase dela processazione dei campioni. Il vuoto, se applicato, non deve superare i 50,79 kPa per evitare danni ai tessuti. Può essere utile anche per rimuovere l'aria intrappolata nei tessuti porosi, accelerando l'impregnazione di tessuti densi e grassi.

\section{Fasi dela processazione dei Tessuti}
\begin{itemize}
    \item \textbf{Fissazione}: stabilizza e indurisce il tessuto con una distorsione minima delle cellule.
    \item \textbf{Disidratazione}: rimozione dell'acqua e del fissativo dal tessuto.
    \item \textbf{Chiarifcazione}: eliminazione delle soluzioni disidratanti per rendere i componenti del tessuto ricettivi al mezzo infiltrante.
    \item \textbf{Infiltrazione}: impregnazione del tessuto con un mezzo di supporto.
    \item \textbf{Inclusione}: orientamento del campione nel mezzo di supporto e solidificazione del preparato.
\end{itemize}

\section{Fissazione}
Preservare le cellule e i componenti tessutali con la minima distorsione possibile è l'obiettivo principale dela processazione dei campioni di tessuto. La fissazione stabilizza le proteine, rendendo la cellula e i suoi componenti resistenti all'autolisi. È fondamentale che la fissazione sia completata prima di iniziare i passaggi successivi.

\section{Disidratazione}
La disidratazione consiste nella rimozione dell'acqua libera e dei fissativi acquosi dai componenti del tessuto. La disidratazione deve essere eseguita gradualmente attraverso una serie di soluzioni a concentrazione crescente. L'etanolo è uno dei reagenti disidratanti più comuni.


\section{Disidratazione}
La prima fase della processazione consiste nella rimozione dell'acqua "libera" non legata e dei fissativi acquosi dai componenti del tessuto. Molti dei reagenti disidratanti sono idrofili, cioè "amano l'acqua", e possiedono forti gruppi polari che interagiscono con le molecole d'acqua nel tessuto tramite legami idrogeno. Altri reagenti influenzano la disidratazione mediante diluizioni ripetute dei fluidi acquosi presenti nei tessuti.

La disidratazione deve essere eseguita gradualmente. Se il gradiente di concentrazione è eccessivo, le correnti di diffusione attraverso le membrane cellulari possono aumentare la possibilità di distorsione cellulare. Per questo motivo, i campioni vengono processati attraverso una serie graduale di reagenti con concentrazioni crescenti. Una disidratazione eccessiva può rendere i tessuti duri, fragili e ristretti. Al contrario, una disidratazione incompleta comprometterà la penetrazione dei reagenti di chiarificazione, lasciando il campione morbido e non ricettivo all'infiltrazione.

Esistono numerosi agenti disidratanti come etanolo, acetone, metanolo, isopropanolo, glicoli e alcol denaturato.

\section{Fluidi Disidratanti}
\subsection{Etanolo (C2H5OH)}
L'etanolo è un liquido limpido, incolore e infiammabile. È idrofilo, miscibile con acqua e altri solventi organici, rapido e affidabile. Oltre ai rischi per la salute umana, l'etanolo è soggetto a tassazione in molti paesi, richiedendo quindi un'attenta gestione dei registri.

Concentrazioni graduali di etanolo vengono utilizzate per la disidratazione; il tessuto è immerso in una soluzione di etanolo al 70\%, seguita da soluzioni al 95\% e al 100\%. L'etanolo assicura una disidratazione totale, il che lo rende il reagente di scelta per la processazione nei casi di microscopia elettronica. Per i tessuti delicati si consiglia di iniziare con una concentrazione di etanolo al 30\%.

\subsection{Spirito Metilato Industriale (Alcol Denaturato)}
Questo fluido ha le stesse proprietà fisiche dell'etanolo. L'alcol denaturato è composto da etanolo con l'aggiunta di metanolo (circa 1\%), isopropanolo o una combinazione di alcoli. Per la processazione dei tessuti, viene utilizzato allo stesso modo dell'etanolo.

\subsection{Metanolo (CH3OH)}
Il metanolo è un fluido limpido, incolore e infiammabile, miscibile con acqua, etanolo e la maggior parte dei solventi organici. È altamente tossico, ma può essere utilizzato come sostituto dell'etanolo nei protocolli di processazione.

\subsection{Propan-2-olo, Alcol Isopropilico (CH3CHOHCH3)}
L'alcol isopropilico è miscibile con acqua, etanolo e la maggior parte dei solventi organici. Viene utilizzato nei protocolli di processazione a microonde. L'alcol isopropilico non provoca un eccessivo indurimento o restringimento del tessuto.

\subsection{Butanolo (C4H9OH)}
Utilizzato principalmente per l'istologia vegetale e animale, il butanolo è un disidratante lento che causa meno restringimento e indurimento del tessuto.

\subsection{Acetone (CH3COCH3)}
L'acetone è un fluido limpido, incolore e infiammabile, miscibile con acqua, etanolo e la maggior parte dei solventi organici. Agisce rapidamente, ma ha una scarsa penetrazione e può causare fragilità nei tessuti se usato a lungo. L'acetone rimuove i lipidi dal tessuto durante la processazione.

\section{Additivi agli Agenti Disidratanti}
Quando viene aggiunto agli agenti disidratanti, il fenolo agisce come ammorbidente per tessuti duri come tendini, unghie, tessuto fibroso denso e masse di cheratina. Si consiglia di aggiungere il 4\% di fenolo a ciascuno stadio con etanolo al 95\%. In alternativa, i tessuti duri possono essere immersi in una miscela di glicerolo e alcol.

\section{Solventi Universali}
I solventi universali non sono più utilizzati per la processazione di routine a causa delle loro proprietà pericolose e devono essere maneggiati con estrema cautela. I solventi universali disidratano e chiarificano i tessuti durante la processazione. Dioxano, butanolo terziario e tetraidrofurano sono considerati solventi universali. Non sono raccomandati per la processazione di tessuti delicati a causa delle loro proprietà indurenti.



\section{Chiarificazione}

Gli agenti di chiarificazione agiscono come intermediari tra i reagenti di disidratazione e quelli di infiltrazione. Essi devono essere miscibili con entrambe le soluzioni. La maggior parte degli agenti chiarificanti sono idrocarburi con indici di rifrazione simili a quelli delle proteine. Quando il disidratante viene completamente sostituito da questi solventi, il tessuto assume un aspetto traslucido, da cui deriva il termine "agente chiarificante".

\subsection{Criteri di selezione per un agente chiarificante}
I criteri per la scelta di un agente chiarificante adatto includono:
\begin{itemize}
    \item Rapida penetrazione nei tessuti;
    \item Rapida rimozione dell'agente disidratante;
    \item Facilità di rimozione mediante cera paraffinica fusa;
    \item Danno minimo ai tessuti;
    \item Bassa infiammabilità;
    \item Bassa tossicità;
    \item Costo ridotto.
\end{itemize}

La maggior parte degli agenti chiarificanti sono liquidi infiammabili e richiedono quindi un uso cauto. Il punto di ebollizione dell'agente chiarificante indica la velocità con cui viene sostituito dalla cera paraffinica fusa. I fluidi con un punto di ebollizione basso tendono a essere sostituiti più rapidamente. La viscosità influisce sulla velocità di penetrazione dell'agente chiarificante. Un'esposizione prolungata agli agenti chiarificanti può rendere il tessuto fragile, pertanto è importante monitorare attentamente il tempo di esposizione per garantire che i blocchi di tessuto denso siano adeguatamente chiarificati e che quelli più piccoli e fragili non vengano danneggiati. Il costo deve essere considerato, specialmente per quanto riguarda lo smaltimento dei reagenti. Poiché la maggior parte degli agenti chiarificanti sono idrocarburi aromatici o alifatici a catena corta, le questioni ambientali devono essere affrontate. La maggior parte delle istituzioni ha una politica per lo stoccaggio, lo smaltimento e i requisiti di sicurezza relativi ai materiali infiammabili utilizzati in laboratorio.

\subsection{Agenti chiarificanti per uso routinario}

\subsubsection{Xilene}
Lo xilene è un liquido infiammabile, incolore, con un caratteristico odore di petrolio o aromatico, miscibile con la maggior parte dei solventi organici e con la cera paraffinica. È adatto per chiarificare blocchi di tessuto inferiori a 5 mm di spessore e sostituisce rapidamente l'alcool dal tessuto. Un'eccessiva esposizione allo xilene durante la processazione può causare indurimento dei tessuti. È comunemente utilizzato nei laboratori di istologia e può essere riciclato.

\subsubsection{Toluene}
Il toluene ha proprietà simili allo xilene, ma è meno dannoso in caso di immersione prolungata del tessuto. Tuttavia, è più infiammabile e volatile rispetto allo xilene.

\subsubsection{Cloroformio}
Il cloroformio è più lento nell'azione rispetto allo xilene, ma provoca meno fragilità nei tessuti. Può essere utilizzato per blocchi di tessuto di spessore maggiore di 1 mm. I tessuti immersi nel cloroformio non diventano traslucidi. È non infiammabile ma altamente tossico e produce gas tossico di fosgene quando riscaldato. Viene utilizzato principalmente nella processazione dei campioni del sistema nervoso centrale.

\subsubsection{Sostituti dello Xilene}
I sostituti dello xilene sono idrocarburi alifatici che esistono in forme a catena lunga e corta. Differiscono per il numero di atomi di carbonio nella catena carboniosa. Gli alifatici a catena corta hanno proprietà di evaporazione simili a quelle dello xilene e non hanno affinità per l'acqua. Gli alifatici a catena lunga non evaporano rapidamente e possono contaminare la cera paraffinica nei processatori di tessuto.

\subsubsection{Oli di agrumi - Reagenti a base di limonene}
I reagenti a base di limonene sono estratti dalle bucce di arancia e limone; sono non tossici e miscibili con l'acqua. Lo smaltimento dipende dai centri di trattamento delle acque e dalle normative locali o nazionali. I principali svantaggi includono la possibilità di sensibilizzazione e il forte odore pungente che può causare mal di testa. Inoltre, piccoli depositi minerali come rame o calcio possono dissolversi e fuoriuscire dai tessuti. Sono estremamente oleosi e non possono essere riciclati.

\section{Reagenti di Infiltrazione e Inclusione}

\subsection{Paraffina}
La cera paraffinica continua a essere il mezzo di infiltrazione e inclusione più popolare nei laboratori di istopatologia. È una miscela di idrocarburi a catena lunga, prodotta dalla fratturazione dell'olio minerale. Le sue proprietà variano a seconda del punto di fusione utilizzato, che può variare tra 47 e 64°C. La cera paraffinica permea il tessuto in forma liquida e si solidifica rapidamente quando raffreddata. Il tessuto viene impregnato con il mezzo, formando una matrice che previene la distorsione della struttura durante il microtomia. 

La cera paraffinica ha una vasta gamma di punti di fusione, il che è importante per l'uso in diverse regioni climatiche del mondo. Per favorire la formazione di nastri desiderabili durante la microtomia, è opportuno scegliere una cera paraffinica di adeguata durezza a temperatura ambiente. Riscaldare la cera paraffinica a temperature elevate può alterarne le proprietà. Le cere paraffiniche con un punto di fusione più alto offrono un supporto migliore per i tessuti più duri.


\section{Mezzi di inclusione alternativi}
In alcuni casi, la paraffina può risultare inadatta per l'inclusione di determinati tipi di tessuti, come nei seguenti esempi:
\begin{itemize}
    \item I reagenti di processazione rimuovono o distruggono i componenti tissutali oggetto dell'indagine, come i lipidi;
    \item È necessario ottenere sezioni più sottili, ad esempio nei linfonodi;
    \item L'uso del calore può danneggiare i tessuti o gli enzimi;
    \item Il mezzo di infiltrazione non è sufficientemente rigido per supportare il tessuto.
\end{itemize}

\subsection{Resina}
La resina è utilizzata esclusivamente come mezzo di inclusione per la microscopia elettronica, in particolare per ottenere sezioni ultrafini ad alta risoluzione e per sezioni di osso non decalcificato.

\subsection{Agar}
Il gel di agar, da solo, non fornisce un supporto sufficiente per sezionare i tessuti. Il suo uso principale è come agente coesivo per piccoli frammenti di tessuto friabili dopo la fissazione, in un processo noto come "doppia inclusione". I frammenti di tessuto vengono immersi in agar fuso, lasciati solidificare e poi tagliati per la processazione di routine. Un metodo superiore e più raffinato consiste nel filtrare il fissativo contenente i frammenti di tessuto attraverso un filtro Millipore con l'ausilio di una pompa di aspirazione. L'agar fuso viene quindi versato con cura nel filtro e, una volta solidificato, il pellet di agar ottenuto viene processato e incluso in paraffina.

\subsection{Gelatina}
La gelatina è utilizzata principalmente per la produzione di sezioni di organi interi con la tecnica Gough-Wentworth e per le sezioni congelate. Il suo utilizzo è raro.

\subsection{Celloidina}
L'uso della celloidina o di LVN (nitrocellulosa a bassa viscosità) è sconsigliato a causa dei requisiti particolari necessari per ospitare i reagenti di processazione e per l'uso limitato che queste sezioni trovano in neuropatologia. L'uso della celloidina è quindi molto raro.

\section{Inclusione in paraffina}
L'inclusione prevede l'incapsulamento di campioni orientati correttamente, dopo essere stati adeguatamente processati, in un mezzo di supporto che fornisca stabilità durante il microtaglio. Il mezzo di inclusione deve riempire la matrice del tessuto, sostenendo i componenti cellulari. Dovrebbe anche fornire elasticità, resistenza alla distorsione durante il taglio, facilitando al contempo il processo di sezionamento.

La maggior parte dei laboratori utilizza centri di inclusione modulari, composti da un dispensatore di paraffina, una piastra fredda e un'area riscaldata per stampi e cassette tissutali. La paraffina viene automaticamente dispensata in uno stampo della dimensione appropriata. Il tessuto viene orientato nello stampo, si applica una cassetta che produce una superficie piana con lati paralleli, e lo stampo viene quindi raffreddato rapidamente, garantendo una struttura cristallina fine e minimizzando gli artefatti durante il taglio.

\section{Processazione automatizzata dei tessuti}
Il principio di base per la processazione dei tessuti prevede lo scambio di fluidi utilizzando una serie di soluzioni per un periodo di tempo predeterminato in un ambiente controllato. Le attrezzature per la processazione dei tessuti sono rimaste relativamente invariate per decenni, ma i recenti progressi includono forni a microonde specializzati, processatori a throughput costante e processatori con retorti multi-sezione.

\subsection{Processatori a microonde}
I forni a microonde progettati specificamente per la processazione dei tessuti sono ora comuni. Questi strumenti riducono i tempi di processazione da ore a minuti, accelerando la diffusione delle soluzioni nei tessuti attraverso l'aumento del calore interno del campione. I forni a microonde da laboratorio hanno un controllo preciso della temperatura, dei tempi e dei sistemi di estrazione dei fumi. Tuttavia, questo metodo richiede attenzione per il controllo della temperatura, e i costi di tali forni possono risultare elevati.

\subsection{Vantaggi delle nuove tecnologie}
I vantaggi principali offerti dalle nuove tecnologie di processazione includono:
\begin{itemize}
    \item Programmi personalizzabili in base al tipo di tessuto;
    \item Schedulazione rapida;
    \item Contenimento dei fluidi e dei fumi;
    \item Reagenti ecologici.
\end{itemize}

\section{Riassunto}
La processazione dei campioni chirurgici e bioptici è un passaggio fondamentale per la preparazione dei tessuti alla valutazione microscopica. Ogni campione viene prima etichettato con un codice unico, garantendo la sua identificazione lungo tutte le fasi del processo. La fissazione stabilizza i tessuti, impedendone la degradazione, mentre la disidratazione elimina l'acqua. Successivamente, il chiarimento prepara i tessuti per l'infiltrazione, che consiste nell'impregnazione con un mezzo di supporto, solitamente paraffina, necessario per il sezionamento.La qualità del campione dipende da fattori come agitazione, calore, viscosità e vuoto, che influenzano la velocità di penetrazione dei reagenti nei tessuti. Alcuni campioni richiedono orientamenti specifici durante l'inclusione, per garantire una corretta visualizzazione microscopica. Le nuove tecnologie, come i processatori a microonde, hanno rivoluzionato il settore, riducendo i tempi di lavorazione e migliorando la qualità delle preparazioni. L'uso di strumenti automatizzati consente una standardizzazione maggiore, personalizzazione dei protocolli e un impatto ambientale ridotto, aumentando efficienza e affidabilità.
