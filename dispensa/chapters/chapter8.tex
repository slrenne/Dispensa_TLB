\chapter{Automazione in anatomia patologica}


\section{Automazione all'interno della metodologia}

L'automazione all'interno della metodologia è già presente in maniera significativa in alcuni settori, in particolare quelli legati alla colorazione in senso lato. Queste procedure richiedono meno manualità, e la variabilità individuale genera grandi differenze nel risultato finale, differenze che non si vogliono avere in presenza di una colorazione istochimica. Un esempio semplice è l'ematossilina-eosina, ma esempi più importanti sono le colorazioni immunoistochimiche, dove la valutazione dell'intensità della colorazione è cruciale.

Un altro aspetto in cui l'automazione risulta essenziale è nelle procedure che richiedono meno manualità, come quei passaggi in cui l'operatore sarebbe esposto a sostanze chimiche pericolose, ad esempio durante la processazione dei campioni. In molti laboratori di anatomia patologica, la processazione è automatizzata, così come la colorazione con l'ematossilina-eosina e le colorazioni immunoistochimiche, specialmente nei laboratori più ricchi a livello globale.

\subsection{Il fattore limitante nella diffusione dell'automazione}

Il fattore limitante nella diffusione dell'automazione è economico. Se un tecnico riesce a svolgere il lavoro di una macchina con la stessa velocità e accuratezza, è difficile immaginare un motivo per utilizzare una macchina, a meno che non sia vantaggioso dal punto di vista dei costi. La macchina dovrebbe avere un costo complessivo così basso (in termini di acquisto, operatività e manutenzione) da rendere l'assunzione di un tecnico svantaggiosa.

Tuttavia, la situazione in anatomia patologica è diversa: in Italia, il numero di casi per anatomia patologica tende ad aumentare, e con casi si intendono il numero di blocchetti e campioni da analizzare. Inoltre, il tipo di colorazioni richieste, sia normali che immunoistochimiche e molecolari, aumenta col tempo.

\subsection{Aumento del numero di esami}

Oggi, esami che una volta non venivano richiesti sono diventati standard. Ad esempio, 50 anni fa, un tumore del colon veniva diagnosticato semplicemente come adenocarcinoma, mentre oggi, oltre alla diagnosi, vengono eseguite colorazioni per il \textit{mismatch repair} (ulteriori 4 blocchetti). Il numero di cassette dedicate ai linfonodi è aumentato, e a volte vengono eseguite colorazioni immunoistochimiche per definire il livello di invasione o la presenza di \textit{budding}. Inoltre, il materiale del paziente può essere ripreso anche dopo anni per esami molecolari.

Questo aumento del numero di esami per caso significa un incremento del carico di lavoro anche se il numero di pazienti non cresce proporzionalmente.

\subsection{Crisi vocazionale e automazione}

Un altro problema è la crisi vocazionale, che colpisce sia il comparto tecnico che medico in anatomia patologica. Questa crisi rende difficile immaginare un futuro senza automazione. Infatti, il personale sanitario non medico, attraverso appositi corsi di studi, si trova a svolgere mansioni che in passato erano di competenza esclusiva dei medici. 

Con la diminuzione del numero di patologi, i tecnici formati per svolgere compiti tipicamente medici occuperanno queste posizioni, diversificando così le proprie mansioni. Di conseguenza, i tecnici si troveranno a svolgere meno attività storicamente legate al proprio ruolo, assumendo nuove responsabilità.

\section{Conclusione}

L'automazione è una necessità reale per il futuro. Nei prossimi paragrafi descriveremo i principi di funzionamento delle macchine che permettono la processazione e la colorazione automatica. Successivamente, esploreremo i principi dei macchinari presenti sul mercato che permettono l'inclusione e il taglio automatico, altre due attività che attualmente si prestano meno all'automazione, ma che, data la situazione di mercato, hanno un margine di crescita nel prossimo futuro.
