\chapter{Automazione in anatomia patologica}


\subsection{Perché automatizzare}

L'automazione sta trasformando profondamente le procedure di laboratorio, in particolare nei settori che richiedono elevata ripetibilità e precisione. Questo fenomeno è particolarmente evidente nelle attività di colorazione, dove la variabilità individuale può influire significativamente sui risultati finali, soprattutto nelle colorazioni istochimiche e immunoistochimiche. La necessità di ridurre tali variabilità spinge verso una maggiore adozione di tecnologie automatizzate, garantendo così risultati più omogenei e affidabili.

\subsection{Automazione e Sicurezza in Laboratorio}

Un altro aspetto cruciale dell'automazione riguarda la sicurezza. In molti passaggi, i tecnici di laboratorio possono essere esposti a sostanze chimiche pericolose, come avviene nella processazione dei campioni. In questi casi, l'automazione riduce il rischio di esposizione e contribuisce a migliorare la sicurezza sul lavoro. La processazione automatizzata è già diffusa nella maggior parte dei laboratori di anatomia patologica, soprattutto nei paesi con risorse economiche maggiori. Questo include anche le procedure di colorazione, come l'ematossilina-eosina e le colorazioni immunoistochimiche.

\section{Analisi Economica dell'Automazione}

\subsection{Costo Opportunità e Efficienza}

Dal punto di vista economico, il principale fattore limitante nella diffusione dell'automazione è legato ai costi. Se un tecnico può svolgere il lavoro di una macchina con velocità e accuratezza comparabili, l'adozione della macchina diventa giustificabile solo se economicamente conveniente. In altre parole, la macchina deve avere un costo d'acquisto, operatività e manutenzione talmente basso da rendere più conveniente la sua adozione rispetto all'assunzione di un tecnico umano. Questo ragionamento si basa sul concetto di \textit{costo opportunità}, ovvero il costo delle risorse impiegate in un'opzione piuttosto che in un'altra.


\subsection{Tecnologie Disruptive e Innovazione}

In questo contesto, il concetto di \textit{disruptive technology} di Christensen diventa fondamentale. Le tecnologie disruptive sono quelle che, pur inizialmente non competitive rispetto alle soluzioni tradizionali in termini di qualità o costo, evolvono rapidamente e cambiano radicalmente il mercato, rendendo obsolete le tecnologie precedenti. L'automazione rappresenta un classico esempio di tecnologia disruptive nei laboratori biomedici: inizialmente vista come un'opzione costosa e non sempre vantaggiosa, oggi si sta rapidamente affermando come standard, soprattutto in risposta alla crescente complessità delle procedure e all'aumento della domanda di esami avanzati.

\section{Aumento della Complessità delle Procedure}

\subsection{Crescente Richiesta di Esami Complessi}

Oggi, rispetto a 50 anni fa, la diagnostica richiede un numero significativamente maggiore di blocchetti e colorazioni per ogni caso clinico. Per esempio, in un tumore del colon, oltre alla diagnosi istologica di adenocarcinoma, vengono richieste ulteriori colorazioni, come quelle per il \textit{mismatch repair}, che aggiungono altri blocchetti da processare. Anche il numero di linfonodi analizzati è aumentato, così come l'utilizzo di colorazioni immunoistochimiche per la valutazione di dettagli diagnostici come il \textit{budding} tumorale o il livello di invasione. Questi esami complessi sono oggi essenziali per una diagnosi accurata e per la personalizzazione del trattamento.

\subsection{Maggiore Complessità = Maggiore Cura nelle Variabili Preanalitiche}

Con l'aumentare della complessità delle tecnologie utilizzate nelle fasi successive del processo diagnostico (ad esempio, esami molecolari sui campioni paraffinati o l'analisi digitale delle immagini), cresce anche l'importanza di un controllo rigoroso delle variabili preanalitiche. Queste includono tutte le fasi che precedono l'analisi del campione, come la fissazione, l'inclusione in paraffina e la preparazione dei vetrini. Ogni errore in queste fasi può compromettere irreparabilmente i risultati delle analisi successive, che sono sempre più sofisticate e sensibili.

Per questo motivo, l'automazione diventa non solo una scelta economica o di efficienza, ma un'utile risposta per garantire l'affidabilità dei risultati in un contesto in cui la variabilità preanalitica può avere un impatto decisivo sulla qualità delle analisi.

\section{Crisi Vocazionale e Trasformazione dei Ruoli}

\subsection{Impatto della Crisi Vocazionale}

Un ulteriore fattore che spinge verso l'automazione è la crisi vocazionale che colpisce sia il comparto tecnico che quello medico, in particolare in settori come l'anatomia patologica. Con un numero decrescente di medici e tecnici qualificati, l'automazione rappresenta una risposta utile per mantenere elevati standard di produttività e qualità nel laboratorio.

\subsection{Nuovi Ruoli per i Tecnici di Laboratorio}

Questa crisi vocazionale porta anche a una ridefinizione dei ruoli nel laboratorio. I tecnici, grazie a percorsi formativi specifici, stanno assumendo competenze e responsabilità che in passato erano appannaggio esclusivo dei medici. In futuro, la diminuzione del numero di patologi potrebbe portare i tecnici a svolgere compiti più complessi, come il campionamento o altre mansioni mediche, lasciando che le macchine si occupino di molte delle attività manuali tradizionalmente associate al loro ruolo.

\section{Conclusioni}

L'automazione è un probabile futuro dei laboratori di anatomia patologica. Non solo per ragioni economiche o di sicurezza, ma soprattutto per garantire la qualità e l'affidabilità dei risultati diagnostici in un contesto in cui le tecnologie a valle stanno diventando sempre più complesse e sensibili alle variabili preanalitiche. Nei prossimi paragrafi, esploreremo il funzionamento delle macchine per la processazione e la colorazione automatica, oltre a discutere delle tecnologie emergenti per l'inclusione e il taglio automatico, settori che mostrano un grande potenziale di crescita nel prossimo futuro.