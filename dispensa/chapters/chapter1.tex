\chapter{Fissazione tessuto istologico}

\section{Introduzione}


\subsection{Importanza della Fissazione dei Tessuti}
La fissazione adeguata dei tessuti per l'esame istologico è fondamentale. Senza un'accurata gestione di questo processo, i test eseguiti in laboratorio rischiano di essere inefficaci o inutili. Negli ultimi cento anni, sono stati sviluppati numerosi tipi di fissativi per preservare la struttura e la funzione biologica dei tessuti. Questi fissativi agiscono mediante vari meccanismi come la formazione di legami covalenti, la disidratazione e l'azione di acidi, sali o calore. I fissativi complessi spesso utilizzano più di uno di questi meccanismi.

\subsection{Vantaggi e Svantaggi dei Fissativi}
Ogni fissativo presenta vantaggi specifici, ma anche numerosi svantaggi. Tra questi si trovano la perdita di molecole, il gonfiore o il restringimento dei tessuti e la variabilità nella qualità delle colorazioni istochimiche e immunoistochimiche. Un problema particolare riguarda l'uso della formaldeide, che può compromettere il riconoscimento antigenico, soprattutto dopo l'inclusione in paraffina. Tuttavia, grazie ai metodi di recupero epitopico indotto dal calore introdotti negli anni '90, molti di questi ostacoli sono stati superati.

\subsection{Perdita di Componenti Molecolari durante la Fissazione}
La fissazione può causare la perdita di componenti solubili, come proteine, lipidi e acidi nucleici, che sono essenziali per mantenere la struttura macromolecolare del tessuto. Se i componenti citoplasmatici vengono persi, la colorazione del tessuto, ad esempio con ematossilina-eosina (H&E), può risultare alterata. Anche le valutazioni immunoistochimiche potrebbero risultare compromesse.

\subsection{Artefatti nei Tessuti Fissati}
La fissazione, inevitabilmente, induce artefatti, come il restringimento o gonfiore dei tessuti, che possono influenzare l’aspetto delle sezioni colorate. Tuttavia, per scopi diagnostici, è importante che questi artefatti siano consistenti e prevedibili, così da non alterare l'interpretazione delle strutture tissutali.

\subsection{Conservazione della Struttura Molecolare}
Uno degli scopi principali della fissazione è preservare le strutture macromolecolari e proteggere i tessuti dalla degradazione. Il fissativo riduce la distruzione enzimatica e protegge i tessuti dagli agenti patogeni. Un fissativo efficace garantisce che le caratteristiche del tessuto possano essere studiate anche a distanza di anni, senza che il tessuto subisca ulteriori danni.

\subsection{ Interazione della Fissazione con Altri Processi}
La fissazione non è un processo isolato: essa interagisce con tutte le fasi successive, dalla disidratazione alla colorazione del tessuto. L’effetto complessivo della fissazione, insieme alla processazione del tessuto, rappresenta un compromesso tra la necessità di preservare la struttura originale e le modifiche inevitabili causate dai diversi processi chimici coinvolti.

\subsection{Fissativo Ideale in Patologia Diagnostica}
Ad oggi, non esiste un fissativo universale ideale. La scelta del fissativo dipende dall’esigenza specifica di evidenziare determinate caratteristiche tissutali. Nella patologia diagnostica, il fissativo più utilizzato è la formalina tamponata al 10\%, che è apprezzata per la sua capacità di preservare le strutture tissutali per molti anni.

\subsection{Caratteristiche di un Buon Fissativo}
Un fissativo efficace deve garantire una colorazione di alta qualità e costante nel tempo, preservando la microarchitettura del tessuto, prevenendo la degradazione enzimatica e minimizzando la diffusione delle molecole solubili. Inoltre, deve garantire la sicurezza d'uso, essere compatibile con le moderne tecnologie e avere un costo sostenibile.

\section{Metodi Fisici di Fissazione}
\subsection{Fissazione per Calore}
La fissazione più semplice è quella per calore. Ad esempio, lessare un uovo provoca la coagulazione delle proteine, permettendo di distinguere chiaramente il tuorlo dall'albume al momento del taglio. Dopo la fissazione per calore, ogni componente diventa meno solubile in acqua rispetto all'uovo fresco. Quando una sezione congelata viene posizionata su un vetrino riscaldato, essa si attacca al vetrino e subisce una parziale fissazione attraverso il calore e la disidratazione. Sebbene una morfologia adeguata possa essere ottenuta lessando i tessuti in soluzione salina normale, in istopatologia il calore è principalmente utilizzato per accelerare altri tipi di fissazione e le fasi di processazione del tessuto.

\subsection{Fissazione con Microonde}
Il riscaldamento a microonde velocizza il processo di fissazione, riducendo i tempi di fissazione di alcuni campioni e sezioni istologiche da oltre 12 ore a meno di 20 minuti. Tuttavia, il riscaldamento dei tessuti in formalina genera una grande quantità di vapori pericolosi. Pertanto, in assenza di una cappa per la fissazione o di un sistema di processazione a microonde progettato per gestire questi vapori, potrebbero sorgere problemi di sicurezza. Recentemente, sono stati introdotti fissativi commerciali a base di glicosale che non producono vapori quando riscaldati a 55°C, offrendo un metodo efficace di fissazione a microonde.

\subsection{Liofilizzazione e Sostituzione a Freddo}
La liofilizzazione è una tecnica utile per lo studio di materiali solubili e piccole molecole. I tessuti vengono tagliati in sezioni sottili, immersi in azoto liquido e l'acqua viene rimossa in una camera a vuoto a −40°C. Successivamente, il tessuto può essere fissato ulteriormente con vapori di formaldeide. Nella sostituzione, i campioni vengono immersi in fissativi a −40°C, come acetone o alcol, che rimuovono lentamente l'acqua attraverso la dissoluzione dei cristalli di ghiaccio, senza denaturare le proteine. L'aumento graduale della temperatura fino a 4°C completa il processo di fissazione. Questi metodi di fissazione sono principalmente utilizzati in ambito di ricerca e sono raramente impiegati nei laboratori clinici.


\section{Fissazione Chimica}
La fissazione chimica utilizza soluzioni organiche o inorganiche per mantenere una adeguata preservazione morfologica. I fissativi chimici possono essere classificati in tre categorie principali: fissativi coagulanti, fissativi a legame incrociato e fissativi composti (Baker, 1958).

\subsection{Fissativi Coagulanti}
Le soluzioni sia organiche che inorganiche possono coagulare le proteine, rendendole insolubili. L'architettura cellulare è mantenuta principalmente da lipoproteine e da proteine fibrose come il collagene; la coagulazione di tali proteine preserva la istomorfologia del tessuto a livello microscopico. Sfortunatamente, poiché i fissativi coagulanti causano flocculazione citoplasmatica e una scarsa conservazione di mitocondri e granuli secretori, questi fissativi non sono utili per l'analisi ultrastrutturale.

\subsection{Fissativi Coagulanti Deidranti}
I fissativi coagulanti più comunemente utilizzati sono alcoli (es. etanolo, metanolo) e acetone. Il metanolo ha una struttura più simile a quella dell'acqua rispetto all'etanolo. Pertanto, l'etanolo compete più fortemente del metanolo nell'interazione con aree idrofobiche delle molecole; la fissazione coagulante inizia a una concentrazione del 50–60\% per l'etanolo, mentre per il metanolo è necessaria una concentrazione dell'80\% o superiore (Lillie e Fullmer 1976). La rimozione e la sostituzione dell'acqua libera nei tessuti da parte di questi agenti hanno diversi effetti potenziali sulle proteine. Le molecole d'acqua circondano le aree idrofobiche delle proteine e, per repulsione, costringono i gruppi chimici idrofobici a entrare in contatto più stretto tra loro, stabilizzando così i legami idrofobici. Rimuovendo l'acqua, il principio opposto indebolisce tali legami. Analogamente, le molecole d'acqua partecipano ai legami idrogeno nelle aree idrofile delle proteine; quindi, la rimozione dell'acqua destabilizza questi legami idrogeno. Insieme, questi cambiamenti agiscono per disturbare la struttura terziaria delle proteine. Inoltre, una volta rimossa l'acqua, la struttura della proteina può diventare parzialmente invertita, con i gruppi idrofobici che si spostano sulla superficie esterna della proteina. Una volta che la struttura terziaria di una proteina solubile è stata modificata, il tasso di ritorno a uno stato solubile più ordinato è lento e la maggior parte delle proteine rimane insolubile anche se riportata in un ambiente acquoso. 

La distruzione della struttura terziaria delle proteine, ovvero la denaturazione, ne cambia le proprietà fisiche, causando potenzialmente insolubilità e perdita di funzione. Anche se la maggior parte delle proteine diventa meno solubile in ambienti organici, fino al 13\% di proteine può andare perso, ad esempio con la fissazione in acetone (Horobin 1982). I fattori che influenzano la solubilità delle macromolecole includono:

\begin{enumerate}
    \item Temperatura, pressione e pH.
    \item Forza ionica del soluto.
    \item La costante di salting-in, che esprime il contributo delle interazioni elettrostatiche.
    \item Le interazioni di salting-in e salting-out.
    \item Il tipo di reagente/i denaturante/i (Herskovits et al. 1970; Horobin 1982; Papanikolau e Kokkinidis 1997; Bhakuni 1998).
\end{enumerate}

L'alcol denatura le proteine in modo diverso, a seconda della scelta e della concentrazione dell'alcol, della presenza di sostanze organiche e inorganiche, e del pH e della temperatura di fissazione. Ad esempio, l'etanolo denatura le proteine più dei fenoli, che a loro volta denaturano più dell'acqua e degli alcoli polivalenti, che denaturano più degli acidi monocarbossilici e di quelli dicarbossilici (Bhakuni 1998).

\subsection{Altri Tipi di Fissativi Coagulanti}
I coagulanti acidi come l'acido picrico e l'acido tricloroacetico modificano le cariche sui gruppi laterali ionizzabili delle proteine, ad esempio (–NH2 → NH3\(^+\)) e (COO\(^{-}\) → COOH), e disturbano i legami elettrostatici e idrogeno. Questi acidi possono anche inserire un anione lipofilo in una regione idrofila, alterando così le strutture terziarie delle proteine (Horobin 1982). L'acido acetico coagula gli acidi nucleici ma non fissa né precipita le proteine; pertanto, viene aggiunto ad altri fissativi per prevenire la perdita di acidi nucleici. L'acido tricloroacetico (Cl3CCOOH) può penetrare nei domini idrofobici delle proteine e l'anione prodotto (–C–COO\(^{-}\)) reagisce con gruppi amminici caricati. Questa interazione provoca la precipitazione delle proteine ed estrae gli acidi nucleici. L'acido picrico o trinitrofenolo si dissolve leggermente in acqua per formare una soluzione acida debole (pH 2.0). Nelle reazioni, forma sali con gruppi basici delle proteine, causando la coagulazione delle proteine. Se la soluzione viene neutralizzata, la proteina precipitata può ridissolversi. La fissazione con acido picrico produce colorazioni più intense, ma le soluzioni a pH basso possono causare idrolisi e perdita di acidi nucleici.

\subsection{Fissativi a Legame Incrociato Non Coagulanti}
Diversi composti chimici sono stati selezionati come fissativi per le loro potenziali azioni di formazione di legami incrociati all'interno e tra proteine e acidi nucleici, così come tra acidi nucleici e proteine. La formazione di legami incrociati potrebbe non essere un meccanismo principale nei tempi di fissazione brevi attualmente utilizzati, e pertanto i "fissativi a legame covalente" potrebbero essere un nome più appropriato per questo gruppo. Esempi includono formaldeide, glutaraldeide e altri aldeidi, come il cloruro di idrato e il glicosale, sali metallici come cloruro mercurico e cloruro di zinco, e altri composti metallici come il tetrossido di osmio. I gruppi aldeidici sono chimicamente e biologicamente reattivi e sono responsabili di molte reazioni istochimiche, ad esempio i gruppi aldeidici liberi possono essere responsabili di reazioni argentaffin (Papanikolau e Kokkinidis 1997).

\subsection{Fissazione con Formaldeide}
La formaldeide nella sua forma tamponata neutra al 10\% (NBF) è il fissativo più comune utilizzato in patologia diagnostica. La formaldeide pura è un vapore che, quando completamente disciolto in acqua, forma una soluzione contenente il 37–40\% di formaldeide; questa soluzione acquosa è conosciuta come "formolina". La "formolina al 10\%" comunemente usata per la fissazione dei tessuti è una soluzione al 10\% di formalina; cioè, contiene circa il 4\% di peso rispetto al volume di formaldeide. Le reazioni della formaldeide con le macromolecole sono numerose e complesse. Fraenkel-Conrat e i suoi collaboratori, utilizzando chimica semplice, hanno identificato meticolosamente la maggior parte delle reazioni della formaldeide con amminoacidi e proteine (French e Edsall 1945; Fraenkel-Conrat e Olcott 1948a, 1948b; Fraenkel-Conrat e Mecham 1949). In una soluzione acquosa, la formaldeide forma idrato di metilene, un glicole metilenico come primo passo nella fissazione (Singer 1962).

\[
\text{H}_2\text{C} = \text{O} + \text{H}_2\text{O}
