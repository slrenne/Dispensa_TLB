\chapter{Fissazione tessuto istologico}

\section{Introduzione}


\subsection{Importanza della Fissazione dei Tessuti}
La fissazione adeguata dei tessuti per l'esame istologico è fondamentale. Senza un'accurata gestione di questo processo, i test eseguiti in laboratorio rischiano di essere inefficaci o inutili. Negli ultimi cento anni, sono stati sviluppati numerosi tipi di fissativi per preservare la struttura e la funzione biologica dei tessuti. Questi fissativi agiscono mediante vari meccanismi come la formazione di legami covalenti, la disidratazione e l'azione di acidi, sali o calore. I fissativi complessi spesso utilizzano più di uno di questi meccanismi.

\subsection{Vantaggi e Svantaggi dei Fissativi}
Ogni fissativo presenta vantaggi specifici, ma anche numerosi svantaggi. Tra questi si trovano la perdita di molecole, il gonfiore o il restringimento dei tessuti e la variabilità nella qualità delle colorazioni istochimiche e immunoistochimiche. Un problema particolare riguarda l'uso della formaldeide, che può compromettere il riconoscimento antigenico, soprattutto dopo l'inclusione in paraffina. Tuttavia, grazie ai metodi di recupero epitopico indotto dal calore introdotti negli anni '90, molti di questi ostacoli sono stati superati.

\subsection{Perdita di Componenti Molecolari durante la Fissazione}
La fissazione può causare la perdita di componenti solubili, come proteine, lipidi e acidi nucleici, che sono essenziali per mantenere la struttura macromolecolare del tessuto. Se i componenti citoplasmatici vengono persi, la colorazione del tessuto, ad esempio con ematossilina-eosina (H\&E), può risultare alterata. Anche le valutazioni immunoistochimiche potrebbero risultare compromesse.

\subsection{Artefatti nei Tessuti Fissati}
La fissazione, inevitabilmente, induce artefatti, come il restringimento o gonfiore dei tessuti, che possono influenzare l’aspetto delle sezioni colorate. Tuttavia, per scopi diagnostici, è importante che questi artefatti siano consistenti e prevedibili, così da non alterare l'interpretazione delle strutture tissutali.

\subsection{Conservazione della Struttura Molecolare}
Uno degli scopi principali della fissazione è preservare le strutture macromolecolari e proteggere i tessuti dalla degradazione. Il fissativo riduce la distruzione enzimatica e protegge i tessuti dagli agenti patogeni. Un fissativo efficace garantisce che le caratteristiche del tessuto possano essere studiate anche a distanza di anni, senza che il tessuto subisca ulteriori danni.

\subsection{ Interazione della Fissazione con Altri Processi}
La fissazione non è un processo isolato: essa interagisce con tutte le fasi successive, dalla disidratazione alla colorazione del tessuto. L’effetto complessivo della fissazione, insieme alla processazione del tessuto, rappresenta un compromesso tra la necessità di preservare la struttura originale e le modifiche inevitabili causate dai diversi processi chimici coinvolti.

\subsection{Fissativo Ideale in Patologia Diagnostica}
Ad oggi, non esiste un fissativo universale ideale. La scelta del fissativo dipende dall’esigenza specifica di evidenziare determinate caratteristiche tissutali. Nella patologia diagnostica, il fissativo più utilizzato è la formalina tamponata al 10\%, che è apprezzata per la sua capacità di preservare le strutture tissutali per molti anni.

\subsection{Caratteristiche di un Buon Fissativo}
Un fissativo efficace deve garantire una colorazione di alta qualità e costante nel tempo, preservando la microarchitettura del tessuto, prevenendo la degradazione enzimatica e minimizzando la diffusione delle molecole solubili. Inoltre, deve garantire la sicurezza d'uso, essere compatibile con le moderne tecnologie e avere un costo sostenibile.

\section{Metodi Fisici di Fissazione}
\subsection{Fissazione per Calore}
La fissazione più semplice è quella per calore. Ad esempio, lessare un uovo provoca la coagulazione delle proteine, permettendo di distinguere chiaramente il tuorlo dall'albume al momento del taglio. Dopo la fissazione per calore, ogni componente diventa meno solubile in acqua rispetto all'uovo fresco. Quando una sezione congelata viene posizionata su un vetrino riscaldato, essa si attacca al vetrino e subisce una parziale fissazione attraverso il calore e la disidratazione. Sebbene una morfologia adeguata possa essere ottenuta lessando i tessuti in soluzione salina normale, in istopatologia il calore è principalmente utilizzato per accelerare altri tipi di fissazione e le fasi di processazione del tessuto.

\subsection{Fissazione con Microonde}
Il riscaldamento a microonde velocizza il processo di fissazione, riducendo i tempi di fissazione di alcuni campioni e sezioni istologiche da oltre 12 ore a meno di 20 minuti. Tuttavia, il riscaldamento dei tessuti in formalina genera una grande quantità di vapori pericolosi. Pertanto, in assenza di una cappa per la fissazione o di un sistema di processazione a microonde progettato per gestire questi vapori, potrebbero sorgere problemi di sicurezza. Recentemente, sono stati introdotti fissativi commerciali a base di glicosale che non producono vapori quando riscaldati a 55°C, offrendo un metodo efficace di fissazione a microonde.

\subsection{Liofilizzazione e Sostituzione a Freddo}
La liofilizzazione è una tecnica utile per lo studio di materiali solubili e piccole molecole. I tessuti vengono tagliati in sezioni sottili, immersi in azoto liquido e l'acqua viene rimossa in una camera a vuoto a --40°C. Successivamente, il tessuto può essere fissato ulteriormente con vapori di formaldeide. Nella sostituzione, i campioni vengono immersi in fissativi a --40°C, come acetone o alcol, che rimuovono lentamente l'acqua attraverso la dissoluzione dei cristalli di ghiaccio, senza denaturare le proteine. L'aumento graduale della temperatura fino a 4°C completa il processo di fissazione. Questi metodi di fissazione sono principalmente utilizzati in ambito di ricerca e sono raramente impiegati nei laboratori clinici.


\section{Fissazione Chimica}
La fissazione chimica utilizza soluzioni organiche o inorganiche per mantenere una adeguata preservazione morfologica. I fissativi chimici possono essere classificati in tre categorie principali: fissativi coagulanti, fissativi a legame incrociato e fissativi composti.

\subsection{Fissativi Coagulanti}
Le soluzioni sia organiche che inorganiche possono coagulare le proteine, rendendole insolubili. L'architettura cellulare è mantenuta principalmente da lipoproteine e da proteine fibrose come il collagene; la coagulazione di tali proteine preserva la istomorfologia del tessuto a livello microscopico. Sfortunatamente, poiché i fissativi coagulanti causano flocculazione citoplasmatica e una scarsa conservazione di mitocondri e granuli secretori, questi fissativi non sono utili per l'analisi ultrastrutturale.

\subsection{Fissativi Coagulanti Deidranti}
I fissativi coagulanti più comunemente utilizzati sono alcoli (es. etanolo, metanolo) e acetone. Il metanolo ha una struttura più simile a quella dell'acqua rispetto all'etanolo. Pertanto, l'etanolo compete più fortemente del metanolo nell'interazione con aree idrofobiche delle molecole; la fissazione coagulante inizia a una concentrazione del 50–60\% per l'etanolo, mentre per il metanolo è necessaria una concentrazione dell'80\% o superiore. La rimozione e la sostituzione dell'acqua libera nei tessuti da parte di questi agenti hanno diversi effetti potenziali sulle proteine. Le molecole d'acqua circondano le aree idrofobiche delle proteine e, per repulsione, costringono i gruppi chimici idrofobici a entrare in contatto più stretto tra loro, stabilizzando così i legami idrofobici. Rimuovendo l'acqua, il principio opposto indebolisce tali legami. Analogamente, le molecole d'acqua partecipano ai legami idrogeno nelle aree idrofile delle proteine; quindi, la rimozione dell'acqua destabilizza questi legami idrogeno. Insieme, questi cambiamenti agiscono per disturbare la struttura terziaria delle proteine. Inoltre, una volta rimossa l'acqua, la struttura della proteina può diventare parzialmente invertita, con i gruppi idrofobici che si spostano sulla superficie esterna della proteina. 



\tiny
\begin{landscape}
\begin{longtable}{|p{3.5cm}|p{3cm}|p{3.5cm}|p{3cm}|p{3cm}|p{3.5cm}|}
\hline
\textbf{Categoria di fissativo} & \textbf{Disidratanti} (Etanolo, Metanolo, Acetone) & \textbf{Aldeidi reticolanti} (Formaldeide, Glutaraldeide) & \textbf{Combinazione cloruro mercurico con formaldeide o acido acetico} (Zenker’s B5) & \textbf{Tetroxido di osmio} & \textbf{Acido picrico più formalina e acido acetico} (Bouin's) \\ \hline \thispagestyle{empty}
\textbf{Effetto sulle proteine} & Precipita senza aggiunta chimica & Reticolanti: aggiunge gruppi idrossimetilici attivi ad ammine, ammidi, alcuni alcoli reattivi e gruppi sulfidrilici. Reticolazione delle catene laterali amminiche/ammidiche o sulfidriliche delle proteine & Additivo più coagulazione & Reticolazione additiva, un po' di estrazione, un po' di distruzione & Coagulante additivo e non additivo, un po' di estrazione \\ \hline
\textbf{mRNA/DNA} & Lieve & Reticolazione lenta; leggera estrazione & Coagulazione & Leggera estrazione & Nessuna azione \\ \hline
\textbf{Lipidi} & Estrazione estensiva & Nessuna azione & Nessuna azione & Resi insolubili dalla reticolazione con doppi legami & Nessuna azione \\ \hline
\textbf{Carboidrati} & Nessuna azione & Nessuna su carboidrati puri; reticolazione delle glicoproteine & Nessuna azione & Leggera ossidazione & Nessuna azione \\ \hline
\textbf{Qualità della colorazione H\&E} & Soddisfacente & Buona & Buona & Scarsa & Buona \\ \hline
\textbf{Effetto sull'ultrastruttura (organelli)} & Distrugge l'ultrastruttura, inclusi mitocondri, proteine, coagulati & Buona (NBF) a eccellente conservazione con glutaraldeide; adeguata a buona in Carson-Millonig's & Scarsa conservazione & Usato per la visualizzazione delle membrane & Scarsa – tende a distruggere le membrane \\ \hline
\textbf{Formulazione abituale} & Soluzione al 70–100\% o in combinazione con altri tipi di fissativi & Formaldeide (37\%) – soluzione acquosa al 10\% tamponata con fosfati a pH 7.2–7.4. Glutaraldeide – al 2\% tamponata a pH 7.4 & Cloruro mercurico combinato con acido acetico o dicromato o formaldeide più acetato & Soluzione all'1\% tamponata a pH 7.4 & Acido picrico acquoso, formalina, acido acetico glaciale \\ \hline
\textbf{Variabili importanti/problemi} & Tempo, spessore del campione – dovrebbe essere usato solo per campioni piccoli o sottili & Tempo, temperatura, pH, concentrazione/spessore del campione & Tossico & Estremamente tossico & Mitocondri e integrità della membrana nucleare distrutti; non appropriato per alcune colorazioni; mordenzante \\ \hline
\textbf{Usi speciali} & Preserva piccole molecole non lipidiche come il glicogeno; preserva l'attività enzimatica & Fissativo generale universale; migliore per ultrastruttura se usato con post-fissazione a tetrossido di osmio & Eccellente per tessuti emopoietici & Visualizzazione ultrastrutturale delle membrane; lipidi in sezioni congelate & Mordente per colorazioni di tessuto connettivo (tricromica) \\ \hline
\end{longtable}
\end{landscape}

\normalsize
Una volta che la struttura terziaria di una proteina solubile è stata modificata, il tasso di ritorno a uno stato solubile più ordinato è lento e la maggior parte delle proteine rimane insolubile anche se riportata in un ambiente acquoso. 

La distruzione della struttura terziaria delle proteine, ovvero la denaturazione, ne cambia le proprietà fisiche, causando potenzialmente insolubilità e perdita di funzione. Anche se la maggior parte delle proteine diventa meno solubile in ambienti organici, fino al 13\% di proteine può andare perso, ad esempio con la fissazione in acetone. I fattori che influenzano la solubilità delle macromolecole includono:
\begin{enumerate}
    \item Temperatura, pressione e pH.
    \item Forza ionica del soluto.
    \item La costante di salting-in, che esprime il contributo delle interazioni elettrostatiche.
    \item Le interazioni di salting-in e salting-out.
    \item Il tipo di reagente/i denaturante/i.
\end{enumerate}

L'alcol denatura le proteine in modo diverso, a seconda della scelta e della concentrazione dell'alcol, della presenza di sostanze organiche e inorganiche, e del pH e della temperatura di fissazione. Ad esempio, l'etanolo denatura le proteine più dei fenoli, che a loro volta denaturano più dell'acqua e degli alcoli polivalenti, che denaturano più degli acidi monocarbossilici e di quelli dicarbossilici.

\subsection{Altri Tipi di Fissativi Coagulanti}
I coagulanti acidi come l'acido picrico e l'acido tricloroacetico modificano le cariche sui gruppi laterali ionizzabili delle proteine, ad esempio (–NH2 → NH3\(^+\)) e (COO\(^{-}\) → COOH), e disturbano i legami elettrostatici e idrogeno. Questi acidi possono anche inserire un anione lipofilo in una regione idrofila, alterando così le strutture terziarie delle proteine. L'acido acetico coagula gli acidi nucleici ma non fissa né precipita le proteine; pertanto, viene aggiunto ad altri fissativi per prevenire la perdita di acidi nucleici. L'acido tricloroacetico (Cl3CCOOH) può penetrare nei domini idrofobici delle proteine e l'anione prodotto (–C–COO\(^{-}\)) reagisce con gruppi amminici caricati. Questa interazione provoca la precipitazione delle proteine ed estrae gli acidi nucleici. L'acido picrico o trinitrofenolo si dissolve leggermente in acqua per formare una soluzione acida debole (pH 2.0). Nelle reazioni, forma sali con gruppi basici delle proteine, causando la coagulazione delle proteine. Se la soluzione viene neutralizzata, la proteina precipitata può ridissolversi. La fissazione con acido picrico produce colorazioni più intense, ma le soluzioni a pH basso possono causare idrolisi e perdita di acidi nucleici.

\subsection{Fissativi a Legame Incrociato Non Coagulanti}
Diversi composti chimici sono stati selezionati come fissativi per le loro potenziali azioni di formazione di legami incrociati all'interno e tra proteine e acidi nucleici, così come tra acidi nucleici e proteine. La formazione di legami incrociati potrebbe non essere un meccanismo principale nei tempi di fissazione brevi attualmente utilizzati, e pertanto i "fissativi a legame covalente" potrebbero essere un nome più appropriato per questo gruppo. Esempi includono formaldeide, glutaraldeide e altri aldeidi, come il cloruro di idrato e il glicosale, sali metallici come cloruro mercurico e cloruro di zinco, e altri composti metallici come il tetrossido di osmio. I gruppi aldeidici sono chimicamente e biologicamente reattivi e sono responsabili di molte reazioni istochimiche, ad esempio i gruppi aldeidici liberi possono essere responsabili di reazioni argentaffini.

\section{Fissazione con Formaldeide}
La formaldeide nella sua forma tamponata neutra al 10\% (NBF) è il fissativo più comune utilizzato in patologia diagnostica. La formaldeide pura è un vapore che, quando completamente disciolto in acqua, forma una soluzione contenente il 37–40\% di formaldeide; questa soluzione acquosa è conosciuta come "formalina". La "formalina al 10\%" comunemente usata per la fissazione dei tessuti è una soluzione al 10\% di formalina; cioè, contiene circa il 4\% di peso rispetto al volume di formaldeide. Le reazioni della formaldeide con le macromolecole sono numerose e complesse.  In una soluzione acquosa, la formaldeide forma idrato di metilene, un glicole metilenico come primo passo nella fissazione.

\[
\text{H}_2\text{C} = \text{O} + \text{H}_2\text{O\textrightarrow HOCH}_2\text{OH}
\]

L'idrato di metilene reagisce con diverse catene laterali delle proteine, formando gruppi laterali idrossimetilici reattivi (–CH2–OH). Se si utilizzano tempi di fissazione relativamente brevi con formalina tamponata neutra al 10\% (da alcune ore a pochi giorni), la formazione dei gruppi laterali idrossimetilici è probabilmente la reazione primaria e caratteristica. La formazione di legami incrociati è piuttosto rara nei tempi di fissazione relativamente brevi attualmente impiegati.

\subsection{Reazioni della Formaldeide con Proteine Nucleari e Acidi Nucleici}
La formaldeide reagisce anche con le proteine nucleari e gli acidi nucleici. Essa penetra tra gli acidi nucleici e le proteine, stabilizzando l'involucro proteico degli acidi nucleici, e modifica anche i nucleotidi reagendo con i gruppi amminici liberi, come avviene con le proteine. Nel DNA libero e non associato, si ritiene che le reazioni di cross-linking inizino in regioni ricche di adenina-timidina (AT), e l'intensità del cross-linking aumenta con l'aumento della temperatura. La formaldeide reagisce anche con i doppi legami C=C e con i legami –SH nei lipidi insaturi, ma non interagisce con i carboidrati.

\subsection{Catene Laterali Più Reattive con la Formaldeide}
Le catene laterali dei peptidi o delle proteine che sono maggiormente reattive con l'idrato di metilene, e che quindi hanno la maggiore affinità per la formaldeide, includono lisina, cisteina, istidina, arginina, tirosina e i gruppi ossidrilici reattivi di serina e treonina.


\subsection{Reversibilità delle reazioni macromolecolari della formaldeide}
I gruppi reattivi della formaldeide possono legarsi ai gruppi idrogeno o tra di loro, formando ponti metilenici. Se la formalina viene rimossa attraverso il lavaggio, i gruppi reattivi possono tornare rapidamente al loro stato originale, anche se i ponti già formati rimarranno intatti.

\subsection{Effetto del lavaggio prolungato}
Un lavaggio di 24 ore rimuove circa la metà dei gruppi reattivi, mentre un lavaggio di 4 settimane ne elimina fino al 90\%. Questo suggerisce che la formazione dei ponti metilenici è un processo piuttosto lento. Nella fissazione rapida utilizzata in patologia diagnostica, la maggior parte della fissazione della formaldeide si arresta con la formazione di gruppi idrossimetilici reattivi.

\subsection{Conservazione a lungo termine in formalina}
Nel lungo periodo, i gruppi reattivi possono ossidarsi in forme più stabili (ad esempio, acidi come -NH-COOH) che non sono facilmente rimovibili con acqua o alcol. Pertanto, riportare un campione in acqua o alcol dopo la fissazione riduce ulteriormente la fissazione, poiché i gruppi reattivi possono invertire il processo e venire rimossi.

\subsection{Importanza dei ponti metilenici}
Si riteneva che i legami crociati fossero essenziali nella fissazione dei tessuti per scopi biologici, ma è probabile che la formazione di gruppi idrossimetilici denaturi effettivamente le macromolecole, rendendole insolubili. Poiché tali esperimenti non sono stati riprodotti, i meccanismi effettivi della fissazione con formaldeide rimangono incerti.

\subsection{Correzione dell'iperfissazione}
L'iperfissazione dei tessuti può essere parzialmente corretta immergendo il tessuto in ammoniaca concentrata e cloralio idrato al 20\% (Lhotka \& Ferreira 1949). Fraenkel-Conrat e i suoi colleghi osservarono che le reazioni di addizione e condensazione della formaldeide con amminoacidi e proteine erano instabili e potevano essere facilmente invertite tramite diluizione o dialisi.

\subsection{Tipi di legami crociati e loro stabilità}
Il principale tipo di legame crociato a breve termine è quello tra un gruppo idrossimetilico su una catena laterale di lisina e arginina, asparagina, glutammina o tirosina. Ogni legame ha un diverso grado di stabilità, che può essere modificato da temperatura, pH e l'ambiente circostante il tessuto (Eltoum et al. 2001b).

\subsection{Saturazione dei tessuti con formalina}
Il tempo necessario per saturare i tessuti umani e animali con gruppi reattivi tramite formalina è di circa 24 ore, ma la formazione di legami crociati può continuare per molte settimane (Helander 1994).

\subsection{Effetto dell'acidità nella formalina}
Quando la formaldeide si dissolve in una soluzione acquosa non tamponata, si forma una soluzione acida (pH 5,0-5,5) a causa della presenza di acido formico nella formaldeide commerciale. La formalina acida reagisce più lentamente con le proteine rispetto alla formalina tamponata neutra (NBF), poiché i gruppi amminici diventano caricati positivamente (ad esempio -N+H3). Tuttavia, la formalina acida preserva meglio il riconoscimento immunitario rispetto alla NBF (Arnold et al. 1996).

\subsection{Uso della formalina acida in immunoistochimica}
Il successo iniziale di Taylor nell'immunoistochimica, nella dimostrazione di immunoglobuline in sezioni di tessuti trattati con paraffina, probabilmente si basava sulla fissazione dei tessuti in formalina acida (Taylor et al. 1974). Tuttavia, l'uso della formalina acida provoca la formazione di un pigmento nero-marrone correlato all'emoglobina degradata, che può essere un problema nei pazienti con malattie del sangue.

\subsection{Effetto della formaldeide su proteine, nucleotidi e lipidi}
La formaldeide preserva principalmente i legami peptidici e la struttura generale degli organelli cellulari. Interagisce con gli acidi nucleici ma ha un effetto minimo sui carboidrati. I lipidi vengono conservati se le soluzioni contengono calcio (Bayliss High \& Lake 1996).


\section{Fattori che Influenzano la Qualità della Fissazione}
\subsection{Buffer e pH}
L'effetto del pH sulla fissazione con formaldeide può essere significativo, a seconda delle applicazioni a cui saranno sottoposti i tessuti. In un ambiente fortemente acido, i gruppi amminici primari (–NH2) attraggono ioni di idrogeno (–NH3+) diventando non reattivi con la formaldeide idratata, e i gruppi carbossilici (–COO--) perdono le loro cariche. Ciò può influenzare la struttura delle proteine. Anche i gruppi idrossilici degli alcoli, come serina e treonina, possono diventare meno reattivi in ambienti acidi. La formazione di gruppi idrossimetilici reattivi e di legami incrociati è ridotta in formaldeide non tamponata al 4\%, che è leggermente acida, poiché i principali legami si formano tra lisina e gruppi amminici liberi delle catene laterali. Alcuni autori hanno suggerito che la formalina non tamponata sia un fissativo migliore rispetto alla formalina tamponata neutra per il riconoscimento immunologico di molti antigeni, specialmente prima degli anni '90, quando i metodi di recupero degli epitopi tramite calore non erano ancora diffusi. Tuttavia, è essenziale evitare ritardi nella fissazione degli antigeni labili come i recettori degli estrogeni durante i test immunoistochimici per biomarcatori clinicamente importanti.

Mentre la formaldeide rimane il metodo raccomandato per preservare in modo ottimale le caratteristiche morfologiche, proteine e acidi nucleici in ambiente clinico, il modo più affidabile per ottenere una fissazione ottimale è tamponare la formalina a un pH tra 7,2 e 7,4. A pH acido, i prodotti metabolici dell'emoglobina vengono modificati chimicamente, formando un pigmento bruno-nero insolubile e birifrangente. Questo pigmento si forma a un pH inferiore a 5,7 e la sua formazione aumenta tra pH 3,0 e 5,0. Sebbene non influenzi solitamente la diagnosi, il pigmento può essere rimosso facilmente con una soluzione alcolica di acido picrico. Per evitare la formazione di pigmento da formalina, si preferisce usare la formalina tamponata neutra.

\subsection{Durata della Fissazione e Dimensione dei Campioni}
Medawar (1941) ha studiato i fattori che influenzano la diffusione del fissativo nei tessuti, scoprendo che la profondità raggiunta è proporzionale alla radice quadrata della durata della fissazione. Questo implica che la fissazione procede lentamente e che il tempo necessario per fissare completamente un campione dipende dalla sua dimensione. Ad esempio, un campione di 10 mm richiederà circa 25 ore per essere fissato completamente. Anche i componenti di un fissativo composto penetrano nei tessuti a velocità diverse, quindi questo effetto si manifesta maggiormente in campioni sottili.

Campioni non fissati devono essere tagliati e non devono superare lo spessore di 0,5 cm. La fissazione di campioni sottili in formalina tamponata neutra può essere completata in 5–6 ore, ma la formazione di legami incrociati in tempi così brevi rimane incerta, e la predominanza di gruppi idrossimetilici potrebbe influenzare l'uso di tecniche molecolari. Studi recenti hanno dimostrato che una fissazione troppo rapida può compromettere la conservazione di antigeni importanti, come il recettore degli estrogeni, nei campioni di tessuto mammario per test immunoistochimici. Per questo motivo, le linee guida raccomandano una fissazione minima di 6–8 ore per i campioni clinici di cancro al seno.

\subsection{Temperatura della Fissazione}
La diffusione delle molecole aumenta con l'aumento della temperatura, e quindi la fissazione con formaldeide avviene più rapidamente a temperature più elevate. Le microonde sono state utilizzate per accelerare la fissazione con formaldeide aumentando sia la temperatura che il movimento molecolare, anche se questo comporta rischi per la sicurezza.

\subsection {Riassunto} 

%}%

Di tutti i fissativi proposti, la formalina tamponata al 10\% rimane la scelta migliore nella maggior parte delle circostanze. È economica, consente al tessuto di rimanere immerso per lunghi periodi senza deteriorarsi ed è compatibile con la maggior parte delle colorazioni speciali, incluse le tecniche immunoistochimiche, purché il tessuto venga posto nel fissativo entro 30 minuti dall'asportazione chirurgica ed evitando un’eccessiva fissazione (oltre 24-48 ore). La "formalina pura" è una soluzione concentrata al 40\% di gas formaldeide in acqua. Una soluzione al 10\% di formalina rappresenta quindi una soluzione al 4\% di formaldeide, ovvero una soluzione 1.3 molare. Se la diluizione finale è mantenuta tra l'8\% e il 12\%, non si notano differenze significative. Tuttavia, se la concentrazione scende sotto il 5\%, la qualità della preparazione ne risente. Ciò può accadere inavvertitamente, soprattutto in luoghi in cui la "formalina pura" viene adulterata con acqua. Rodriguez-Martinez e colleghi hanno sviluppato una formula semplice per verificare la diluizione finale del fissativo e correggerla, se necessario, misurando la densità specifica del liquido.

Contrariamente a quanto si crede, il restringimento dei tessuti durante la fissazione con formalina è minimo. Qualsiasi restringimento che si verifica è dovuto alle proprietà contrattile del campione, come dimostrato dal fatto che tende a verificarsi subito dopo l'escissione, prima della fissazione, ed è correlato alla quantità di tessuto contrattile presente. Un esempio evidente è lo strato muscolare esterno del tratto gastrointestinale. È stato calcolato che i segmenti del colon-retto si riducono del 57\% della loro lunghezza in vivo. Gran parte di questo restringimento può essere evitato fissando il campione su una tavola di sughero prima della fissazione.

Il liquido di Zenker (che contiene cloruro mercurico) è un eccellente fissativo, tra i migliori mai sviluppati per la microscopia ottica, ma è costoso, richiede una gestione accurata dello smaltimento del mercurio e necessita di grande attenzione ai tempi di fissazione e alle procedure di lavaggio per rimuovere i precipitati di mercurio. Questo fissativo o il sublimate formalinato di acetato di sodio ('B-5') è spesso utilizzato per biopsie di rene, midollo osseo, linfonodi e testicolo. Il fissativo di Bouin (che contiene acido picrico) è stato raccomandato in particolare per le biopsie testicolari, ma il liquido di Zenker produce preparati quasi identici. Bouin, Zenker e B-5 sono eccellenti fissativi per il lavoro di routine e per la maggior parte delle colorazioni immunoistochimiche, ma la conservazione degli acidi nucleici è molto scarsa.

Il fissativo di Carnoy è una miscela di etanolo, cloroformio e acido acetico glaciale. Mentre fissa i tessuti, dissolve la maggior parte dei grassi, proprietà utile per identificare i linfonodi in campioni di resezioni radicali.

Con l'introduzione di tecniche speciali nella diagnostica patologica, si sono cercati fissativi compatibili sia con la gestione routinaria che con l'applicazione delle tecniche specifiche. Quando la microscopia elettronica era di moda, fu proposto un "fissativo universale", composto da una miscela di paraformaldeide commerciale al 4\% e glutaraldeide all'1\% in tampone neutro. Con l'avvento delle tecniche immunoistochimiche, furono introdotti fissativi specifici per questo scopo. Attualmente, con l'interesse per le tecniche molecolari, si stanno facendo sforzi per sviluppare fissativi in grado di preservare il più possibile la quantità e l'integrità degli acidi nucleici presenti. Un esempio è il fissativo a base di etanolo al 70\%, che, a differenza della formalina, non crea legami incrociati e provoca poche alterazioni chimiche al DNA, tranne un collasso reversibile. Un altro fissativo proposto è il methacam, una soluzione di Carnoy in cui il metanolo sostituisce l'etanolo. Mentre la ricerca del fissativo universale continua, l'approccio più sensato è quello di trattare il tessuto secondo le raccomandazioni specifiche per la tecnica utilizzata.

Quando si utilizza la formalina, il volume del fissativo deve essere almeno 10 volte quello del tessuto. Il contenitore dovrebbe avere un'apertura sufficientemente ampia da permettere la facile rimozione del tessuto una volta indurito dalla fissazione. Il fissativo dovrebbe circondare il campione su tutti i lati. I campioni grandi che galleggiano nel fissativo devono essere coperti da uno spesso strato di garza. Nel caso di campioni grandi, piatti e pesanti che riposano sul fondo del contenitore, la garza dovrebbe essere posta tra il fondo del contenitore e il campione.

La fissazione può essere eseguita a temperatura ambiente o, nel caso di campioni grandi, a 4°C. Il tessuto non dovrebbe essere congelato una volta immerso nella soluzione fissativa, poiché si formerebbero distorsioni dovute ai cristalli di ghiaccio.
